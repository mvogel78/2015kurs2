\documentclass[11pt]{article}

\oddsidemargin=0in
\evensidemargin=0in
\textwidth=6.3in
\topmargin=-0.5in
\textheight=9in

\parindent=0in
\pagestyle{empty}

\begin{document}
\section{Exercises}
\begin{enumerate}
\item  For the Cars93 (MASS) data set, answer the following:
  \begin{enumerate}
  \item For MPG. highway modeled by Horsepower, find the simple regression
    coefficients. What is the predicted mileage for a car with 225 horsepower?
  \item Fit the linear model with MPG. highway modeled by Weight. Find the predicted highway mileage of a 6,400 pound HUMMER H2 and a 2,524 pound MINI Cooper.
  \item Fit the linear model Max .Price modeled by Min .Price. Why might you expect the slope to be around 1 ?
  \end{enumerate}
  Can you think of any other linear relationships among the variables?
\item  For the data set MLBattend (UsingR) concerning major league baseball
  attendance, fit a linear model of attendance modeled by wins. What is the predicted increase in attendance if a team that won 80 games last year wins 90 this year?
\item  People often predict children’s future height by using their 2-year-old height. A common rule is to double the height. Table 10.2 contains data for eight people’s heights as 2-year-olds and as adults. Using the data, what is the predicted adult height for a 2-year-old who is 33 inches tall?

  \begin{tabular}{l c c c c c c c c}
    \hline
    Age 2 (inch) & 39 & 30 & 32 & 34 & 35 & 36 & 36 & 30 \\
    \hline
    Adult (in.) & 71 & 63 & 63 & 67 & 68 & 68 & 70 & 64 \\
    \hline
  \end{tabular}
\item With the \texttt{rmr} (ISwR) data set, plot metabolic rate versus body weight. Fit a linear regression model to the relation. According to the fitted model, what is the predicted metabolic rate for a body weight of 70 kg? Give a
  95\% confidence interval for the slope of the line.
\item In the \texttt{juul} (ISwR) data set, fit a linear regression model for the square root of the IGF-I concentration versus age to the group of subjects over 25 years old.
\item In the \texttt{malaria} data set, analyze the log-transformed antibody level
  versus age. Make a plot of the relation. Do you notice anything peculiar?
\end{enumerate}

\section{Extractor Functions}
\begin{itemize}
\item \texttt{print(m)} simple display
\item \texttt{summary(m)} summary information about the regression
\item \texttt{coef(m)} returns coefficients
\item \texttt{confint(m)} returns confidence intervals for the coefficients
\item \texttt{fitted(m)} returns a vector of fitted values
\item \texttt{resid(m)} returns a vector of residuals
\item \texttt{plot(m)} produces various diagnostic plots based on residuals
\item \texttt{anova(m)} returns various sum of squares
\item \texttt{predict(m, newdata)} predicts the response for new values of the explanatory variables
\item \texttt{deviance(m)} residual sum of squares
\item \texttt{df.residual(m)} for the residual degrees of freedom
\item \texttt{vcov(m)} variance-covariance matrix
\end{itemize}


\section{Example}
\begin{verbatim}
> m <- lm(bweight ~ hyp + gestwks, data=births)
> summary(m)

Call:
lm(formula = bweight ~ hyp + gestwks, data = births)

Residuals:
     Min       1Q   Median       3Q      Max 
-1711.04  -283.13    -9.86   283.92  1361.22 

Coefficients:
             Estimate Std. Error t value Pr(>|t|)    
(Intercept) -4285.002    349.322 -12.267   <2e-16 ***
hyphyper     -143.675     58.820  -2.443   0.0149 *  
gestwks       192.238      8.956  21.465   <2e-16 ***
---
Signif. codes:  0 ‘***’ 0.001 ‘**’ 0.01 ‘*’ 0.05 ‘.’ 0.1 ‘ ’ 1

Residual standard error: 447.5 on 487 degrees of freedom
  (10 observations deleted due to missingness)
Multiple R-squared:  0.5132,	Adjusted R-squared:  0.5112 
F-statistic: 256.7 on 2 and 487 DF,  p-value: < 2.2e-16

> coef(m)
(Intercept)    hyphyper     gestwks 
 -4285.0021   -143.6749    192.2384 
> ggplot(births, aes(x = gestwks,y = bweight)) +
+     geom_point() +
+     geom_abline(intercept = coef(m)[1],
+                 slope = coef(m)[3],colour="blue") +
+     geom_abline(intercept = coef(m)[1] + coef(m)[2],
+                 slope = coef(m)[3],colour="red")
Warnmeldung:
Removed 10 rows containing missing values (geom_point).   
\end{verbatim}


\section{Access Effects More comfortable}
\begin{itemize}
\item the package \texttt{effects} contains a nice function to make effects easier to understand
\item the main function \texttt{Effects} takes the name of one or more covariate and a model as argument and shows the estimates for different level of the covariates
\item below you find a example for the model above
\end{itemize}
> Effect("gestwks",m)

 gestwks effect
gestwks
       25        30        35        40 
 500.1388 1461.3306 2422.5224 3383.7142 
> Effect("hyp",m)

 hyp effect
hyp
  normal    hyper 
3158.824 3015.149 
> Effect(c("gestwks","hyp"),m)

 gestwks*hyp effect
       hyp
gestwks   normal    hyper
     25  520.957  377.282
     30 1482.149 1338.474
     35 2443.341 2299.666
     40 3404.532 3260.857
\begin{verbatim}
  
\end{verbatim}
\end{document}
