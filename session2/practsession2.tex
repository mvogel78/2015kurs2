\documentclass[12pt]{article}
\usepackage{geometry}
\geometry{left=1.2in,top=1in,right=1.2in,bottom=1.5in} %margins
\parindent0mm
\parskip2mm
\begin{document}

\title{Practicals Part 1}
\author{}
\date{\today}
\maketitle

\section{Prepare!}
\begin{enumerate}
\item download and save the session2 folder for today's lesson
\item Run RStudio or Deducer and change your working directory to this copied directory
\item Try to use R as a calculator: What is greater $e^{\pi}$ or $\pi^{e}$.
\item Try the graphical demo using \texttt{demo(graphics)} at the commend line.
\item Look at the objects created by this demo by the function \texttt{ls()}. Please do NOT omit the parenthesis!
\item Delete all objects with \texttt{rm(list=ls())}
\end{enumerate}


\section{Commands you will need}
PS: you may not really \textit{need} all of them but \textit{all of them} will be helpful
\begin{itemize}
\item \texttt{load()}
\item \texttt{c()}
\item \texttt{seq()}
\item \texttt{rep()}
\item \texttt{length()}
\end{itemize}

\section{Indexing}
\begin{enumerate}
\item Create a vector w with components 1, -1, 2, -2
\item Display this vector
\item Obtain a description of w using str()
\item Create the vector w2 as w+1, and display it.
\item Create the vector v with components (0, 1, 5, 10, 15, ... , 75) using c() and/or seq().
\item Find the length of this vector.
\end{enumerate}


\subsection{Displaying and changing parts of a vector (indexing)}
First try to understand the following commands: (you can input the vector via the keyboard or load it - it is contained in the session2 folder)
\begin{verbatim}
> x <- c(2, 7, 0, 9, 10, 23, 11, 4, 7, 8, 6, 0)
> x[4]
> x[3:5]
> x[c(1, 5, 8)]
> x[x > 10]
> x[(1:6) * 2]
> x[x == 0] <- 1
> x
> ifelse(round(x/2) == x/2, "even", "odd")        ## this is an extra
\end{verbatim}

Now try the following (for the modifying parts, first try to display):
\begin{enumerate}
\item Display every third element in x
\item Display elements that are less than 10, but greater than 4
\item Modify the vector x, replacing by 10 all values that are greater than 10
\item Modify the vector x, multiplying by 2 all elements that are smaller than 5
\item Create a new vector y with elements 0,1,0,1, . . . (12 elements) and a vector z that equals x when y=0 and 3x when y=1. (You can do it using ifelse, but there are other possibilities)
\end{enumerate}

\subsection{Displaying and changing parts of a data frame (indexing part2)}
Load the data \texttt{session2/presidential.rdata}. Have a look at the \texttt{presidential} data set. 

\begin{enumerate}
\item Display only lines containing Republican presidents
\item Display only lines containing Democratic presidents
\item Display only lines with presidents reigned longer than 3 but less than 6 years
\item type and understand the following commands
\begin{verbatim}
> table(presidential$party)
> presidential$durmax <- presidential$duration==8
> pt <- table(presidential$party,presidential$durmax)
> pt
> plot(pt)
> by(presidential$duration,presidential$party,summary)
\end{verbatim}
\end{enumerate}



\end{document}
\end
