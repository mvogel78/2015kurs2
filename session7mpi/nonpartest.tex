% July 2015
% Autor: Mandy Vogel
% non-parametric tests

\documentclass[xcolor={table},handout]{beamer}
\usepackage{amssymb,amsmath}
%\documentclass[xcolor={table}]{beamer}
% \usetheme[backgroundimagefile=mathe]{diepen}
\usetheme{Singapore}
\useoutertheme{miniframes}

%\setbeamerfont{block title}{size=\small,series=\bfseries}
%\setbeamerfont{block body}{size=\footnotesize}

% \usecolortheme{beetle}
\usepackage{linkimage}

%\usepackage{handoutWithNotes}
%\pgfpagesuselayout{2 on 1}[a4paper,border shrink=5mm]
%\pgfpagesuselayout{3 on 1 with notes}[a4paper,border shrink=5mm]

\begin{document}

\title{Alternatives to the t-test}   
\author{Mandy Vogel} 
\date{\today}

\AtBeginSection{
  \begin{frame}<beamer>{Table of Contents}
    \tableofcontents[currentsection]
  \end{frame}}

\begin{frame}
\titlepage
\end{frame}

\begin{frame}{Table of Contents}
\frametitle{Table of Contents}\tableofcontents
\end{frame}

\section{Permutation Tests}
\subsection{Permutation Tests}
\begin{frame}\frametitle{Permutation Tests}
\begin{itemize}
\item does not rely on a assumed a priori distribution
\item instead a empirical distribution is created from randomization of observed data
\item robust against deviations from normality
\item sensitive to differences in treatment variances
\item scope of inference is limited to the sample (importance has been questioned by a number of authors)
\end{itemize}
\end{frame}

\begin{frame}[fragile]\frametitle{Excurse - \texttt{tapply()}}
\begin{itemize}
\item \texttt{tapply()} allows you to create tables (hence the "t") of the value of a function on subgroups defined by its second argument, which can be a factor or a list of factors.
\item e.g. in the \texttt{sleep} data frame, we can  summarize \texttt{extra} (increase in hours of sleep) classify by \texttt{group} (drug given) as follows:
\begin{exampleblock}{Input/Output}\small
\begin{verbatim}
> tapply(X = sleep$extra, INDEX = sleep$group, FUN = mean)
   1    2 
0.75 2.33 
> tapply(sleep$extra,sleep$group,mean)
   1    2 
0.75 2.33 
\end{verbatim}
\end{exampleblock}
\item \texttt{lapply()} is a member of the apply functions 
\end{itemize}
\end{frame}


\begin{frame}[fragile]\frametitle{Excurse - \texttt{replicate()}}
\begin{itemize}
\item the function \texttt{replicate()} is also a member of the apply family
\item it is used for repeated evaluation of an expression (usually in the context of random number generation)
\item example: roll 6 dice and sum them up 10000 times
\end{itemize}
\end{frame}

\begin{frame}[fragile]\frametitle{Excurse - \texttt{replicate()}}
\begin{exampleblock}{Input/Output}\footnotesize
\begin{verbatim}
> sample(1:6,size = 6, replace = T)
[1] 2 6 2 3 5 3
> sum(sample(1:6,size = 6, replace = T))
[1] 19 ## not the same sample as above
> dice <- replicate(10000,sum(sample(1:6,size = 6,replace = T)))
> table(dice)
dice
  6   7   8   9  10  11  12  13  14  15  16  17  18  19 ... 
  1   1   3   7  16  49  80 163 274 357 494 602 723 842 ...
 26  27  28  29  30  31  32  33  34 
483 356 270 170 108  63  33  11   4 
\end{verbatim}
\end{exampleblock}
\end{frame}


\begin{frame}\frametitle{Permutation Tests - Example}
  \begin{itemize}
  \item we use the \texttt{sleep} data set
  \item first we calculate the difference of the group means using the functions \texttt{lapply()} and \texttt{diff()}
  \item we use the absolute difference (this corresponds to a two-sided test)
  \end{itemize}
\end{frame}


\begin{frame}[fragile]\frametitle{Permutation Tests - Example}
\begin{exampleblock}{Input/Output}\small
\begin{verbatim}
## use tapply to calculate the group means (not necessary)
> tapply(sleep$extra,sleep$group,FUN = mean)
   1    2 
0.75 2.33 
> ## calculate the difference between the group means
> ## (not necessary)
> diff(tapply(sleep$extra,sleep$group,FUN = mean))
   2 
1.58 
> ## calculate the absolute difference (two-sided test)
> orig.diff <- abs(diff(tapply(sleep$extra,sleep$group,mean)))
> orig.diff
   2 
1.58 
\end{verbatim}
\end{exampleblock}
\end{frame}


\begin{frame}[fragile]\frametitle{Permutation Tests - Example}
\begin{exampleblock}{Input/Output}\small
\begin{verbatim}
> ## do it a 10000 times
> res <- replicate(10000, 
+    abs(diff(tapply(sleep$extra,sample(sleep$group),mean))))
> sum(orig.diff <= res)/10000 ## p-value
[1] 0.0724
\end{verbatim}
\end{exampleblock}
\end{frame}


\begin{frame}[fragile]\frametitle{Permutation Tests - Example}
\begin{exampleblock}{Input/Output}\small
\begin{verbatim}
> ## compare with t-test
> t.test(sleep$extra ~ sleep$group)

	Welch Two Sample t-test

data:  sleep$extra by sleep$group
t = -1.8608, df = 17.776, p-value = 0.07939
alternative hypothesis: true difference in means is not equal to 0
95 percent confidence interval:
 -3.3654832  0.2054832
sample estimates:
mean in group 1 mean in group 2 
           0.75            2.33 
\end{verbatim}
\end{exampleblock}
\end{frame}

\subsection{Rank-Based Permutation Tests}
\begin{frame}\frametitle{Rank-Based Permutation Tests}
\begin{itemize}
\item requiring a single null distribution given a particular sample size
\item much less sensitive to outliers compared to parametric methods
\item scope of inference is generally less considered an issue
\item slightly less powerful than parametric methods if their parametric assumptions hold
\item computational problems of ties
\item rank transformation throws out a large amount of information
\item also sensitive to heteroscedasticity
\end{itemize}
\end{frame}

\begin{frame}\frametitle{Rank-Based Permutation Tests}
  \begin{alertblock}{Warnings}
    \begin{itemize}
    \item these tests are not free from assumptions
    \item hypotheses statements will generally differ from parametric tests
    \end{itemize}
  \end{alertblock}
\end{frame}


\begin{frame}\frametitle{Rank-Based Permutation Tests - Example}

  \begin{itemize}
  \item consider to hypothesized populations $X_1$ and $X_2$
  \item assume two observations for $X_1$ and three observations for $X_2$ 
  \item one-tailed test: H_0: $X_1 \ge X_2$ versus H_A: $X_1 < X_2$
  \item in absence of ties we have $(n_1 + n_2)!/(n_1!n_2!)=10$ possiple ranks for $X_1$ or $X_2$
  \item what is the smallest possible p-value?
  \end{itemize}
\end{frame}

\section{Wilcoxon Test}
\begin{frame}\frametitle{Wilcoxon tests}
    \begin{itemize}
    \item the Wilcoxon tests the location of the median
    \item it is a non-parametric alternative to Student's t test
    \item it is based on the ranks -- and really simple, e.g. for the two sample test: sort your data, give them ranks, sum up the ranks by group, take the smaller sum and look in a table for the appropriate row/column (with ties are dealt with by averaging the appropriate ranks) 
    \item in R it is (not very surprisingly) \texttt{wilcox.test()}
    \item there is also a one and a two sample and a paired version
    \end{itemize}
\end{frame}

\begin{frame}[fragile]\frametitle{Wilcoxon Test}
\begin{itemize}
\item the non-parametric test is much more appropriate when the errors are not normal, 
\item can be more powerful if the distribution is strongly skewed by the presence of outliers
\item typically the t-test will give the lower p-value, so the Wilcoxon test is said to be conservative: if a difference is significant under a Wilcoxon test it would have been even more significant under a t-test.
\end{itemize}
\end{frame}


\begin{frame}[fragile]\frametitle{Wilcoxon Signed Rank Test}
\begin{itemize}
\item method for one sample or two dependent samples
\item calculate the differences between the pairs of observations (or between values and hypothesized value)
\item let $n$ be the number of non-zero differences
\item rank the absolute values of the $n$ differences
\item reassign the signs from step 1
\item $T_+$ is the sum of the positive signed ranks, while $T_-$ is the sum of the negative ranks
\item for a two tailed test the minimum of these two is taken as test statistic, in a upper-tailed $T_-$, in a lower-tailed $T_+$
\end{itemize}
\end{frame}


\begin{frame}[fragile]\frametitle{Wilcoxon Signed-Rank Test}\footnotesize
\begin{verbatim}
> pre.test <- c(17,12,20,12,20,21,23,10,15,17,18,18)
> post.test <- c(19,25,18,18,26,19,27,14,20,22,16,18)
> wilcox.test(pre.test,post.test,paired = T)

	Wilcoxon signed rank test with continuity correction

data:  pre.test and post.test
V = 7.5, p-value = 0.02527
alternative hypothesis: true location shift is not equal to 0

Warning messages:
1: In wilcox.test.default(pre.test, post.test, paired = T) :
  kann bei Bindungen keinen exakten p-Wert Berechnen
2: In wilcox.test.default(pre.test, post.test, paired = T) :
  kann den exakten p-Wert bei Nullen nicht berechnen
\end{verbatim}
\end{frame}



\begin{frame}\frametitle{Exercise}
  \begin{enumerate}
  \item load the data frame normtemp, which is contained in the UsingR package; it contains the body temperature of several individuals, the gender and the heart rate
  \item test if the temperature is different in male (coded as 1) and female (coded as 2), use the appropriate test.
  \item test again, compare the results of the t test and the wilcoxon.
  \item plot the respective boxplots!
  \end{enumerate}
\end{frame}

\begin{frame}[allowframebreaks]\frametitle{Exercise}
  \begin{enumerate}
  \item load the \texttt{UsingR} package
  \item The Simple data set \texttt{iq} contains simulated scores on a hypothetical IQ test. What analysis is appropriate for measuring the center of the distribution? Why?
  \item The Simple data set \texttt{slc} contains data on red blood cell sodium-lithium countertransport activity for 190 individuals. Describe the shape of the distribution, estimate the center, state what is an appropriate measure of center for this data.
  \item  Load the Simple data set \texttt{vacation}. This gives the number of paid holidays and vacation taken by workers in
    the textile industry.
    \begin{enumerate}
    \item Is a test for $\bar{y}$ appropriate for this data?
    \item Does a t-test seem appropriate?
    \item If so, test the null hypothesis that $\mu = 24$. 
    \end{enumerate}
  \item  Repeat the above for the Simple data set \texttt{smokyph}. This data set measures pH levels for water samples in the Great Smoky Mountains. Use the waterph column \texttt{smokyph[[’waterph’]]} to test the null hypothesis that
    $\mu = 7$. 
  \end{enumerate}
\end{frame}

\end{document}
