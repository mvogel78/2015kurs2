\section{Combining Data Frames}
\subsection{\texttt{rbind()}}
\begin{frame}[fragile]\frametitle{\texttt{rbind()}}
\begin{itemize}
\item \texttt{rbind()} can be used to combine two dataframes (or matrices) in the sense of adding rows, the column names and types must be the same for the two objects
  \begin{exampleblock}{Input/Output}\small
\begin{verbatim}
> x <- data.frame(id=1:3,score=rnorm(3))
> y <- data.frame(id=13:15,score=rnorm(3))
> rbind(x,y)
  id       score
1  1  0.71121163
2  2 -0.62973249
3  3  1.17737595
4 13 -0.45074940
5 14 -0.01044197
6 15 -1.05217176
\end{verbatim}
  \end{exampleblock}
\end{itemize}
\end{frame}

\subsection{\texttt{cbind()}}
\begin{frame}[fragile]\frametitle{\texttt{cbind()}}
\begin{itemize}
\item \texttt{cbind()} can be used to combine two dataframes (or matrices) in the sense of adding columns, the number of rows must be the same for the two objects
  \begin{exampleblock}{Input/Output}\small
\begin{verbatim}
> cbind(x,y)
  id      score1      score2     score3
1  1  0.11440705  0.14536778 -1.1773241
2  2 -1.62862651  0.02020604  0.5686415
3  3  0.05335811  0.25462270  0.8844987
4  4 -0.19931734  0.15625511  0.9287316
5  5 -1.15217836 -1.79804503 -0.7550234
\end{verbatim}
  \end{exampleblock}
\item it is not recommended to use \texttt{cbind()} to combining data frames
\end{itemize}
\end{frame}


\subsection{\texttt{merge()}}
\begin{frame}[fragile,allowframebreaks]\frametitle{\texttt{merge()}}
\begin{itemize}
\item \texttt{merge()} is the command of choice for merging or joining data frames
\item it is the equivalent of join in sql
\item there are four cases
  \begin{itemize}
  \item inner join
  \item left outer join
  \item right outer join
  \item full outer join
  \end{itemize}
\end{itemize}
  \begin{exampleblock}{Input/Output}\footnotesize
\begin{verbatim}
> (d1 <- data.frame(id=LETTERS[c(1,2,3)],day1=sample(10,3)))
  id day1
1  A    3
2  B    4
3  C    5
> (d2 <- data.frame(id=LETTERS[c(1,3,5,6)],day2=sample(10,4)))
  id day2
1  A    7
2  C   10
3  E    3
4  F    6
\end{verbatim}
  \end{exampleblock}
\end{frame}


\begin{frame}[fragile]\frametitle{\texttt{inner join}}
\begin{itemize}
\item inner join means: keep only the cases present in both of the data frames
  \begin{exampleblock}{Input/Output}\small
\begin{verbatim}
> merge(d1,d2)
  id day1 day2
1  A    3    7
2  C    5   10
\end{verbatim}
  \end{exampleblock}
\end{itemize}
\end{frame}

\begin{frame}[fragile]\frametitle{\texttt{left outer join}}
\begin{itemize}
\item left outer join means: keep all cases of the left data frame no matter if they are present in the right data frame (\texttt{all.x=T})
  \begin{exampleblock}{Input/Output}\small
\begin{verbatim}
> merge(d1,d2,all.x = T)
  id day1 day2
1  A    3    7
2  B    4   NA
3  C    5   10
\end{verbatim}
  \end{exampleblock}
\end{itemize}
\end{frame}


\begin{frame}[fragile]\frametitle{\texttt{right outer join}}
\begin{itemize}
\item right outer join means: keep all cases of the right data frame no matter if they are present in the left data frame (\texttt{all.y=T})
  \begin{exampleblock}{Input/Output}\small
\begin{verbatim}
> merge(d1,d2,all.y = T)
  id day1 day2
1  A    3    7
2  C    5   10
3  E   NA    3
4  F   NA    6
\end{verbatim}
  \end{exampleblock}
\end{itemize}
\end{frame}


\begin{frame}[fragile]\frametitle{\texttt{full outer join}}
\begin{itemize}
\item full outer join means: keep all cases of both data frames (\texttt{all=T})
  \begin{exampleblock}{Input/Output}\small
\begin{verbatim}
> merge(d1,d2,all = T)
  id day1 day2
1  A    3    7
2  B    4   NA
3  C    5   10
4  E   NA    3
5  F   NA    6
\end{verbatim}
  \end{exampleblock}
\end{itemize}
\end{frame}

\begin{frame}[fragile]\frametitle{\texttt{merge()}}
\begin{itemize}
\item if not stated otherwise R uses the intersect of the names of both data frames, in our case only \textit{id}
\item you can specify these columns directly by \texttt{by=c("colname1","colname2")} if the columns are named identical or
\item using\\ \texttt{by.x=c("colname1.x","colname2.x"),
by.y=c("colname1.y","colname2.y")} if they have different names in the data frames
\end{itemize}
\end{frame}

\begin{frame}[fragile,allowframebreaks]\frametitle{\texttt{merge() - Exercise}}
On the course website you find an excel file \emph{nhanes.xlsx} containing 4 tables: demographics, body measurements, blood pressure, and physical activity. Each of the tables is a part of the nhanes 2011-2012 \url{http://wwwn.cdc.gov/nchs/nhanes/search/nhanes11_12.aspx}. The file \emph{codebook.txt} contains a short version of a codebook.
\end{frame}

\begin{frame}[fragile,allowframebreaks]\frametitle{\texttt{merge() - Exercise}}
\begin{itemize}
\item try to load the package \texttt{readxl}. If it is not already installed, install it first.
\item use the \texttt{read\_excel()} command to read in all 4 data sets.\footnotesize
  \begin{exampleblock}{Input/Output}\small
\begin{verbatim}
> ## excel_sheets() given a filename returns all 
> ## available sheets 
> excel_sheets("nhanes1112.xlsx")
[1] "demographics" "bp"           "physactivity" "bodymeas"    
> ## read_excel() takes the filename and the sheet name 
> ## (or position)
> ## and reads in the data 
> demogr <- read_excel("nhanes1112.xlsx","demographics")
\end{verbatim}
  \end{exampleblock}
\end{itemize}
\end{frame}

\begin{frame}[fragile,allowframebreaks]\frametitle{\texttt{merge() - Exercise}}
\begin{itemize}
\item use \texttt{merge()} to combine these data frames twice
  \begin{enumerate}
  \item keep all the rows while merging
  \item keep only the rows which are present in all of the data sets
  \end{enumerate}
  how many rows do both data frame have?
\end{itemize}
\end{frame}

