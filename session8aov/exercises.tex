\documentclass[11pt]{article}
\usepackage{hyperref}

\oddsidemargin=0in
\evensidemargin=0in
\textwidth=6.3in
\topmargin=-0.5in
\textheight=9in

\parindent=0in
\pagestyle{empty}

\begin{document}
\section{Exercises}
\subsection{Combining Data Frames}
On the course website you find an excel file \emph{nhanes.xlsx} containing 4 tables: demographics, body measurements, blood pressure, and physical activity. Each of the tables is a part of the nhanes 2011-2012 \url{http://wwwn.cdc.gov/nchs/nhanes/search/nhanes11_12.aspx}. The file \emph{codebook.txt} contains a short version of a codebook.
\begin{itemize}
\item try to load the package \texttt{readxl}. If it is not already installed, install it first.
\item use the \texttt{read\_excel()} command to read in all 4 data sets.\footnotesize
\begin{verbatim}
> ## excel_sheets() given a filename returns all 
> ## available sheets 
> excel_sheets("nhanes1112.xlsx")
[1] "demographics" "bp"           "physactivity" "bodymeas"    
> ## read_excel() takes the filename and the sheet name 
> ## (or position)
> ## and reads in the data 
> demogr <- read_excel("nhanes1112.xlsx","demographics")
\end{verbatim}
\item use \texttt{merge()} to combine these data frames twice
  \begin{enumerate}
  \item keep all the rows while merging
  \item keep only the rows which are present in all of the data sets
  \end{enumerate}
  how many rows do both data frame have?
\end{itemize}

\subsection{Anova}
  \begin{enumerate}
  \item Look at the help of the TukeyHSD function. What is its purpose? 
  \item Execute the code of the example near the end of the help page, interpret the results!
  \item install and load the granovaGG package (a package for visualization of ANOVAs), load the \texttt{arousal} data frame and use the \texttt{stack()} command to bring the data in the long form. Do a anova analysis using the \texttt{granova.1w()}. Is there a difference between at least 2 of the groups? If indicated do a post-hoc test.
  \item Visualize your results using boxplots
  \end{enumerate}
  \begin{enumerate}
  \item Look at the help of the TukeyHSD function. What is its purpose? 
  \item Execute the code of the example near the end of the help page, interpret the results!
  \item install and load the granovaGG package (a package for visualization of ANOVAs), load the \texttt{arousal} data frame and use the \texttt{stack()} command to bring the data in the long form. Do a anova analysis. Is there a difference at least 2 of the groups? If indicated do a post-hoc test.
  \end{enumerate}
\end{document}
