\section{ANOVA}
\begin{frame}\frametitle{ANOVA}
  \begin{itemize}  
  \item a technique we use when all explanatory variables are categorical (factor)
  \item generalization of the t-test
  \item if there is one factor with three or more levels we use one-way ANOVA (only two levels: t-test should be preferred, would give exactly the same answer since with 2 levels $F=t^2$)
  \end{itemize}
\end{frame}

\begin{frame}\frametitle{ANOVA}
  \begin{itemize}  
  \item for more factors there there is two-way, three-way anova 
  \item central idea is to compare two or more means by comparing variances
  \item the statistical model $$y_{ij} = \mu_i + \epsilon_{ij}$$ where the error terms are independent an $\epsilon_{ij} \sim \mathcal{N}(0,\sigma)$ 
  \end{itemize}
\end{frame}

\subsection{Data}
\begin{frame}[fragile]\frametitle{The Garden Data}
A data frame with 14 observations on 2 variables. 
\begin{center}
\rowcolors{1}{gray!10}{gray!30}
\begin{tabular}{@{} >{\ttfamily}r l}
  ozone: & athmospheric ozone concentration               \\
  garden: & garden id                              \\
\end{tabular}

\vspace*{1cm}

\begin{table}[ht]
\small
\centering
\begin{tabular}{rllllllllllllll}
  \hline
 & 1 & 2 & 3 & 4 & 5 & 6 & 7 & 8 & 9 & 10 & 11 & 12 & 13 & 14 \\ 
  \hline
ozone &  9 &  7 &  6 &  8 &  5 & 11 &  9 & 11 &  9 &  6 & 10 &  8 &  8 & 12 \\ 
  garden & a & a & a & b & a & b & b & b & b & a & b & a & a & b \\ 
   \hline
\end{tabular}
\end{table}
\end{center}
From: Michael Crawley, The R-Book

\end{frame}

\subsection{Sums of Squares}
\begin{frame}\frametitle{Total Sum of Squares}
\tikzstyle{na} = [baseline=-.5ex]
  \begin{itemize}
  \item  we plot the values in order they are measured
  \end{itemize}
\begin{center}
\includegraphics[width=11cm]{TSS1.png}
\end{center}
\end{frame}

\begin{frame}\frametitle{Total Sum of Squares}
  \begin{itemize}
  \item  there is a lot of scatter, indicating that the variance in ozone is large
  \item to get a feel for the overall variance we plot the overall mean (8.5) and indicate each of the residuals by a vertical line
  \end{itemize}
\end{frame}

\begin{frame}\frametitle{Total Sum of Squares}
\begin{center}
\includegraphics[width=11cm]{img/TSS.png}
\end{center}
\end{frame}

\begin{frame}\frametitle{Total Sum of Squares}
  \begin{itemize}
  \item we refer to this overall variation as the \emph{total sum of squares, SSY or TSS} 
$$ SSY = \sum(y-\bar{y})^2$$
  \end{itemize}
\end{frame}

\begin{frame}\frametitle{Total Sum of Squares}
  \begin{itemize}
  \item in this case $$SSY = 55.5$$
  \end{itemize}
   \tikzstyle{background grid}=[draw, black!50,step=.5cm]
        \begin{tikzpicture}
            % Put the graphic inside a node. This makes it easy to place the
            % graphic and to draw on top of it. 
            % The above right option is used to place the lower left corner
            % of the image at the (0,0) coordinate. 
            \node [inner sep=0pt,above right] 
                {\includegraphics[width=11cm]{TSS2.png}};
            \filldraw[fill=blue!40,opacity=.5] (5.7,1.1) ellipse (5cm and 0.5cm);
%%            \fill (21.5,1.8) circle (2pt);
            % define destination coordinates
        \end{tikzpicture}
\end{frame}

\begin{frame}\frametitle{Group Means}
  \begin{itemize}
  \item now instead of fitting the overall mean, let us fit the individual garden means
  \end{itemize}
\begin{table}[ht]
\centering
\begin{tabular}{lcc}
  \hline
garden & a & b \\ 
  mean &  7 & 10 \\ 
   \hline
\end{tabular}
\end{table}
\end{frame}

\begin{frame}\frametitle{Group Means}
\tikzstyle{na} = [baseline=-.5ex]
  \begin{columns}
    \begin{column}{0.7\paperwidth}
      \begin{itemize}
      \item[]  \tikz[na] \coordinate (s-A); Garden A
      \end{itemize}
    \end{column}
    \begin{column}{0.4\paperwidth}
      \begin{itemize}
      \item[] Garden B \tikz[na] \coordinate (s-B);
      \end{itemize}
    \end{column}
  \end{columns}
%%  \tikzstyle{background grid}=[draw, black!50,step=.5cm]
        \begin{tikzpicture}
            % Put the graphic inside a node. This makes it easy to place the
            % graphic and to draw on top of it. 
            % The above right option is used to place the lower left corner
            % of the image at the (0,0) coordinate. 
            \node [inner sep=0pt,above right] 
            {\includegraphics[width=10cm]{img/ESS.png}};
            % define destination coordinates
            \path (9.75,4.2) coordinate (B)
            (.6,2.3) coordinate (A);
        \end{tikzpicture}

% define overlays
% Note the use of the overlay option. This is required when 
% you want to access nodes in different pictures.
\begin{tikzpicture}[overlay]
        \path[->,red,thick] (s-B) edge [bend left] (B);
        \path[->,black,thick] (s-A) edge [bend right] (A);
\end{tikzpicture}

\end{frame}


\begin{frame}\frametitle{Group Means}
  \begin{itemize}
  \item now we see that the mean ozone concentration is substantially higher in garden B
  \item the aim of ANOVA is to determine 
    \begin{itemize}
    \item whether it is significantly higher \emph{or}
    \item whether this kind of difference could come by chance alone
    \end{itemize}
  \end{itemize}
\end{frame}

\begin{frame}\frametitle{Error Sum of Squares}
\emph{ When the means are significantly different then the sum of squares computed from the individual garden means will be smaller than the sum of squares computed from the overall mean. }
  \begin{itemize}
  \item we define the new sum of squares as the \emph{error sum of squares} (error in the sense of 'residual')
$$ SSE = \sum(y_{garden A}-\bar{y}_{garden A})^2+\sum(y_{garden B}-\bar{y}_{garden B})^2$$
  \end{itemize}
\end{frame}

\begin{frame}\frametitle{Total Sum of Squares}
  \begin{itemize}
  \item in this case $$SSE = 24.0$$
  \end{itemize}
\begin{center}
\includegraphics[width=10cm]{img/ESS2.png}
\end{center}
\end{frame}


\begin{frame}\frametitle{Treatment Sum of Squares}
\begin{itemize}
  \item then the component of the variation that is explained by the difference of the means is called the \emph{treatment sum of squares} SSA
  \item analysis of variance is based  on the notion that we break down the total sum of squares into useful and informative components
$$SSY=SSE+SSA$$ where
    \begin{itemize}
      \item SSA = explained variation
      \item SSE = unexplained variation
    \end{itemize}

  \end{itemize}
\end{frame}

\begin{frame}\frametitle{ANOVA table}
\begin{center}
\small
%% \rowcolors{1}{gray!10}{gray!30}
\begin{tabular}{@{} >{\ttfamily}l cccr}
\hline
Source & Sum of squares & Degrees of freedom & Mean square & F ratio \\
\hline
Garden &  $31.5$ & $1$ &  $31.5$ &  $15.75$\\
Error &  $24.0$ & $12$ &  $s^2=2.0$ &  \\
Total & $55.5$ & $13$ & & \\   \hline
\end{tabular}
\end{center}
\end{frame}


\begin{frame}[fragile]\frametitle{ANOVA}
  \begin{itemize}
  \item now we need to test whether an F ratio of 15.75 is large or small
  \item we can use a table or software package
  \item I use here software to calculate the cumulative probability
  \end{itemize}
\begin{verbatim}
> 1 - pf(15.75,1,12)
[1] 0.001864103
\end{verbatim}
\end{frame}

\begin{frame}\frametitle{ANOVA}
\begin{center}
\includegraphics[width=10cm]{img/fdens.png}
\end{center}
\end{frame}


\begin{frame}[fragile]\frametitle{ANOVA in R}
  \begin{itemize}
  \item in R we use the \texttt{lm()} or the \texttt{aov()} command and
  \item the formula syntax \texttt{a \sim  b}
  \item we assign this to an variable
 \end{itemize}
\end{frame}


\begin{frame}[fragile]\frametitle{ANOVA in R}
\begin{verbatim}
mm <- lm(ozone ~ garden, data=oneway)
mm

Call:
lm(formula = ozone ~ garden, data = oneway)

Coefficients:
(Intercept)      gardenb  
          7            3  
\end{verbatim}
\end{frame}


\begin{frame}[fragile]\frametitle{ANOVA in R}
\footnotesize
\begin{verbatim}
> summary(mm)

Call:
lm(formula = ozone ~ garden, data = oneway)

Residuals:
   Min     1Q Median     3Q    Max 
    -2     -1      0      1      2 

Coefficients:
            Estimate Std. Error t value Pr(>|t|)    
(Intercept)   7.0000     0.5345  13.096 1.82e-08 ***
gardenb       3.0000     0.7559   3.969  0.00186 ** 
---
Signif. codes:  0 ‘***’ 0.001 ‘**’ 0.01 ‘*’ 0.05 ‘.’ 0.1 ‘ ’ 1

Residual standard error: 1.414 on 12 degrees of freedom
Multiple R-squared:  0.5676,	Adjusted R-squared:  0.5315 
F-statistic: 15.75 on 1 and 12 DF,  p-value: 0.001864
\end{verbatim}
\end{frame}

\begin{frame}[fragile]\frametitle{ANOVA in R}
\footnotesize
\begin{verbatim}
> anova(mm)
Analysis of Variance Table

Response: ozone
          Df Sum Sq Mean Sq F value   Pr(>F)   
garden     1   31.5    31.5   15.75 0.001864 **
Residuals 12   24.0     2.0                    
---
Signif. codes:  0 ‘***’ 0.001 ‘**’ 0.01 ‘*’ 0.05 ‘.’ 0.1 ‘ ’ 1
\end{verbatim}
\end{frame}



\begin{frame}[fragile,allowframebreaks]\frametitle{ANOVA in R}
\footnotesize
\begin{verbatim}
> m2 <- aov(ozone ~ garden, data=oneway)
> m2
Call:
   aov(formula = ozone ~ garden, data = oneway)

Terms:
                garden Residuals
Sum of Squares    31.5      24.0
Deg. of Freedom      1        12

Residual standard error: 1.414214
Estimated effects may be unbalanced
> summary(m2)
            Df Sum Sq Mean Sq F value  Pr(>F)   
garden       1   31.5    31.5   15.75 0.00186 **
Residuals   12   24.0     2.0                   
---
Signif. codes:  0 ‘***’ 0.001 ‘**’ 0.01 ‘*’ 0.05 ‘.’ 0.1 ‘ ’ 1
> summary.lm(m2)

Call:
aov(formula = ozone ~ garden, data = oneway)

Residuals:
   Min     1Q Median     3Q    Max 
    -2     -1      0      1      2 

Coefficients:
            Estimate Std. Error t value Pr(>|t|)    
(Intercept)   7.0000     0.5345  13.096 1.82e-08 ***
gardenb       3.0000     0.7559   3.969  0.00186 ** 
---
Signif. codes:  0 ‘***’ 0.001 ‘**’ 0.01 ‘*’ 0.05 ‘.’ 0.1 ‘ ’ 1

Residual standard error: 1.414 on 12 degrees of freedom
Multiple R-squared:  0.5676,	Adjusted R-squared:  0.5315 
F-statistic: 15.75 on 1 and 12 DF,  p-value: 0.001864

> summary(m2)
            Df Sum Sq Mean Sq F value  Pr(>F)   
garden       1   31.5    31.5   15.75 0.00186 **
Residuals   12   24.0     2.0                   
---
Signif. codes:  0 ‘***’ 0.001 ‘**’ 0.01 ‘*’ 0.05 ‘.’ 0.1 ‘ ’ 1
\end{verbatim}
\end{frame}



\begin{frame}\frametitle{ANOVA Assumptions}
  \begin{alertblock}{Central Assumptions}
  \begin{itemize}
  \item independed, normal distributed errors
  \item equality of variances (homogeneity)
  \end{itemize}
  \end{alertblock}
\end{frame}


\begin{frame}[allowframebreaks]\frametitle{Welch ANOVA}
\begin{itemize}
\item generalization of the Welch t-test
\item tests whether the means of the outcome variables are different across the factor levels
\item assumes sufficiently large sample (greater than 10 times the number of groups in the calculation, groups of size one are to be excluded)
\item sensitive to the existence of outliers (only few are allowed)
\item the r command is \texttt{oneway.test()}
\item non-parametric alternative \texttt{kruskal.test()}
\end{itemize}
\end{frame}


\begin{frame}\frame{Exercises}
  \begin{enumerate}
  \item Look at the help of the TukeyHSD function. What is its purpose? 
  \item Execute the code of the example near the end of the help page, interpret the results!
  \item install and load the granovaGG package (a package for visualization of ANOVAs), load the \texttt{arousal} data frame and use the \texttt{stack()} command to bring the data in the long form. Do a anova analysis. Is there a difference at least 2 of the groups? If indicated do a post-hoc test.
  \item Visualize your results
  \end{enumerate}
\end{frame}



\begin{frame}[allowframebreaks,fragile]\frametitle{Exercises - Solutions}
  \begin{enumerate}
  \item Look at the help of the TukeyHSD function. What is its purpose? 
  \item Execute the code of the example near the end of the help page, interpret the results!
  \item install and load the granovaGG package (a package for visualization of ANOVAs), load the \texttt{arousal} data frame and use the \texttt{stack()} command to bring the data in the long form. Do a anova analysis. Is there a difference at least 2 of the groups? If indicated do a post-hoc test.\scriptsize
\begin{verbatim}
> require(granovaGG)
> data(arousal)
> datalong <- stack(arousal)
> m1 <- aov(values ~ ind, data = datalong)
> summary(m1)
            Df Sum Sq Mean Sq F value   Pr(>F)    
ind          3  273.4   91.13   10.51 4.17e-05 ***
Residuals   36  312.3    8.68                     
---
Signif. codes:  0 ‘***’ 0.001 ‘**’ 0.01 ‘*’ 0.05 ‘.’ 0.1 ‘ ’ 1
> TukeyHSD(m1)
  Tukey multiple comparisons of means
    95% family-wise confidence level

Fit: aov(formula = values ~ ind, data = datalong)

$ind
                  diff           lwr        upr     p adj
Drug.A.B-Drug.A   3.54  -0.007542384  7.0875424 0.0506601
Drug.B-Drug.A    -0.45  -3.997542384  3.0975424 0.9860554
Placebo-Drug.A   -3.84  -7.387542384 -0.2924576 0.0296168
Drug.B-Drug.A.B  -3.99  -7.537542384 -0.4424576 0.0223986
Placebo-Drug.A.B -7.38 -10.927542384 -3.8324576 0.0000137
Placebo-Drug.B   -3.39  -6.937542384  0.1575424 0.0654726
\end{verbatim}\normalsize
  \item Visualize your results\scriptsize
\begin{verbatim}
> ggplot(datalong,aes(x=ind,y=values)) + 
+        geom_boxplot()  
\end{verbatim}
\begin{center}
\includegraphics[width=10cm]{img/aovgr1.png}
\end{center}
\begin{verbatim}
> granovagg.1w(datalong$values,group = datalong$ind)

By-group summary statistics for your input data (ordered by group means)
     group group.mean trimmed.mean contrast variance standard.deviation
4  Placebo      20.43        20.30    -3.65     5.83               2.41
3   Drug.B      23.82        23.85    -0.26     7.50               2.74
1   Drug.A      24.27        24.45     0.19     7.89               2.81
2 Drug.A.B      27.81        27.52     3.73    13.49               3.67
  group.size
4         10
3         10
1         10
2         10

Below is a linear model summary of your input data

Call:
lm(formula = score ~ group, data = owp$data)

Residuals:
   Min     1Q Median     3Q    Max 
-5.910 -2.015 -0.075  1.885  6.290 

Coefficients:
              Estimate Std. Error t value Pr(>|t|)    
(Intercept)    24.2700     0.9314  26.057  < 2e-16 ***
groupDrug.A.B   3.5400     1.3172   2.688  0.01083 *  
groupDrug.B    -0.4500     1.3172  -0.342  0.73461    
groupPlacebo   -3.8400     1.3172  -2.915  0.00608 ** 
---
Signif. codes:  0 ‘***’ 0.001 ‘**’ 0.01 ‘*’ 0.05 ‘.’ 0.1 ‘ ’ 1

Residual standard error: 2.945 on 36 degrees of freedom
Multiple R-squared:  0.4668,	Adjusted R-squared:  0.4223 
F-statistic:  10.5 on 3 and 36 DF,  p-value: 4.173e-05

\end{verbatim}
\begin{center}
\includegraphics[width=10cm]{img/aovgr2.png}
\end{center}
  \end{enumerate}
\end{frame}
