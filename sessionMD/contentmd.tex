\section{What is Markdown}
\label{md1}

\begin{frame}\frametitle{Markdown}
  And yet -- another language
  \begin{itemize}
  \item Markdown is a lightweight markup language
  \item plain text formatting syntax designed
  \item it can be converted to HTML and many other formats
  \item it is very easy, so we will begin immediately
  \end{itemize}
\end{frame}


\begin{frame}\frametitle{Configuration RStudio (for PDF)}
  \begin{itemize}
  \item go to \textit{Tools}
  \item then \textit{Global Options}
  \item \textit{Sweave}
  \item choose \textit{knitr} for \texttt{Weave Rnw files using}
  \item press \textit{Apply} and \textit{OK}
  \end{itemize}
\end{frame}


\begin{frame}\frametitle{The first R Markdown file}
  \begin{itemize}
  \item in RStudio go to the \textit{New file} menu
  \item choose \textit{R Markdown}
  \item in the dialog box type a title and your name in the respective fields
  \item choose \textit{HTML} 
  \item choose \textit{Document} from the list in the left-hand side
  \item press ok
  \item now save the file via the menu or clicking on the disc symbol
  \end{itemize}
\end{frame}


\begin{frame}\frametitle{The First Page}
  \begin{itemize}
  \item what you get is a little example Markdown document
  \item and you can produce a nice formatted html page by clicking on \textit{knit HTML}
  \end{itemize}
\end{frame}


\begin{frame}\frametitle{Change Output Format}
  \begin{itemize}
  \item you can change the export format to word or pdf by clicking on the arrow next to  \textit{knit HTML}
  \item for producing PDFs you need a working \LaTeX installation
  \item there are \LaTeX distributions for Windows, Linux and MacOS
  \item \url{http://latex-project.org/ftp.html}
  \item \LaTeX is big, so the installation takes a while
  \end{itemize}
\end{frame}

\begin{frame}[fragile]\frametitle{The Example Document}
  If you have a look at the markdown in the example
  \begin{itemize}
  \item the header information are enclosed by \verb|---|
\begin{verbatim}
---
title: "My first markdown document"
author: "Mandy"
date: "22. August 2015"
output: word_document
---
\end{verbatim}
  \end{itemize}
\end{frame}

\begin{frame}[fragile]\frametitle{The Example Document}
  If you have a look at the markdown in the example
  \begin{itemize}
\item keywords: \texttt{title, author,} and \texttt{date}
  \item the output format is specified by \texttt{output} (automatically done by RStudio)
  \item for producing PDFs the dot contained in the date should be removed or escaped
\begin{verbatim}
---
title: "My first markdown document"
author: "Mandy"
date: "22 August 2015"
output: word_document
---
\end{verbatim}

  \end{itemize}
\end{frame}



\begin{frame}[fragile]\frametitle{The Example Document}
  \begin{itemize}
  \item in line 8 you see how to insert a link using angle brackets\small
\begin{verbatim}
This is an R Markdown document. Markdown is a simple
formatting syntax for authoring HTML, PDF, and MS Word
documents. For more details on using R Markdown see
<http://rmarkdown.rstudio.com>.
\end{verbatim}
  \end{itemize}
\end{frame}


\begin{frame}[fragile]\frametitle{Links}
  \begin{itemize}
  \item similarly you can create links, e.g. the word \texttt{RStudio} with a link to the Rstudio website: 
\begin{verbatim}
This service is provided
by [RStudio](https://www.rstudio.com/)
\end{verbatim}
  \end{itemize}
\end{frame}


\begin{frame}[fragile]\frametitle{The Example Document}
  \begin{itemize}
  \item one or two stars (or underscores) can be used to produce italics or bold like in the next line
\begin{verbatim}
When you click the **Knit** button a document will
be generated that includes both content as well as
the output of any embedded R code chunks within the
document. You can embed an R code chunk like this:
\end{verbatim}

  \end{itemize}
\end{frame}

\begin{frame}[fragile]\frametitle{The Example Document}
  \begin{itemize}
  \item Try to add the following lines and rerun \textit{Knit HTML}
    
\begin{verbatim}
*italics* or _italics_

**bold** or __bold__

superscript^2^

~~strike through~~
\end{verbatim}
  
  \end{itemize}
\end{frame}


\begin{frame}[fragile]\frametitle{Code Chunks}
  \begin{itemize}
  \item the next you see is
    \small
\begin{verbatim}
```{r}
summary(cars)
```
\end{verbatim}
    \normalsize
  \item if you look in the produced html page you see that code as well as the result
  \item you can insert such chunks by typing
        \small
\begin{verbatim}
```{r}

```
\end{verbatim}
        \normalsize
    or by clicking \textit{Insert Chunks} in the \textit{Code} menu or the \textit{Chunks} drop down in the upper right corner of the editor window
  \item or by typing \texttt{Ctrl+Alt+i}
  \end{itemize}
\end{frame}


\begin{frame}[fragile]\frametitle{Inline Code}
  \begin{itemize}
  \item inline code is insert with a pair of single back tick:
\begin{verbatim}
The **cars** data frame contains `r nrow(cars)`
rows and `r ncol(cars)` columns.
\end{verbatim}
  \end{itemize}
\end{frame}


\begin{frame}[fragile]\frametitle{Pander}
  \begin{itemize}
  \item So. The result is OK, but not pretty.
  \item there is a R package called pander
  \item install the package
  \item and add the following lines to your Rmd file
    
\begin{verbatim}
```{r}
require(pander)
pander(summary(cars))
```
\end{verbatim}
  
  \end{itemize}
\end{frame}


\begin{frame}\frametitle{Pander}
  \begin{itemize}
  \item the pander package contains mainly just one command: \texttt{pander()}
  \item \texttt{pander()} tries to format the output of R commands nicely for plotting
  \end{itemize}

\end{frame}

\begin{frame}[fragile]\frametitle{Pander}
Exercise:
  \begin{enumerate}
  \item add a section to your Rmd file. You can do this using \verb|#|
    
\begin{verbatim}
# Linear Model
\end{verbatim}
    
  \item add a new code chunk to your Rmd file
  \item use \texttt{lm()} to build a model of \texttt{dist} dependend on \texttt{speed} with the data from the \texttt{cars} data frame
  \item use \texttt{summary()} and \texttt{pander()} to add the output to your document
  \end{enumerate}
\end{frame}


\begin{frame}[fragile]\frametitle{Pander}
  \begin{itemize}
  \item if you do not want to include the R code itself in your document add the option \texttt{echo = F} to the markdown file
\begin{verbatim}
```{r  echo = F}
m <- lm(dist ~ speed, data = cars)
pander(summary(m))
```
\end{verbatim}
  \end{itemize}
\end{frame}


\begin{frame}[fragile]\frametitle{Hide R Output}
  \begin{itemize}
  \item you can hide R code and results by using \texttt{results = 'hide'}
\begin{verbatim}
```{r  echo = F, results = 'hide'}
m <- lm(dist ~ speed, data = cars)
pander(summary(m))
```
\end{verbatim}
  \item the output of messages, warnings and errors can be suppressed by
    \begin{itemize}
    \item \texttt{message = F}
    \item \texttt{warning = F} and
    \item \texttt{error= F} respectively
    \end{itemize}
  \end{itemize}
\end{frame}

  
\begin{frame}[fragile]\frametitle{Plots}
Exercise: The next part of the examples contains a graphic
  \begin{enumerate}
  \item insert a section above the plot
\begin{verbatim}
# Graphics
\end{verbatim}
  \item and a subsection
\begin{verbatim}
## using plot
\end{verbatim}
    
  \item now insert a new subsection and try to build the same plot using \texttt{ggplot()}
  \item do not forget to load the package
  \item for the scatter plot you need the \texttt{geom\_point()}
  \item add the line corresponding to the linear model (\texttt{geom\_line()} and you have to specify the method!)
  \end{enumerate}
\end{frame}


\begin{frame}[fragile]\frametitle{Plots}
  \begin{itemize}
  \item you can change the width and height of the plot by using \texttt{fig.width} and \texttt{fig.width} (not supported for word export)
  \item \texttt{fig.align} can be used to change the alignment
\begin{verbatim}
```{r fig.align='center', fig.width=6}
require(ggplot2)
ggplot(cars,aes(x=speed, y=dist)) +
  geom_point() +
  geom_smooth(method = "lm")
```
\end{verbatim}
  \end{itemize}
\end{frame}



\begin{frame}[fragile]\frametitle{Lists}
  \begin{itemize}
  \item there are ordered and unordered lists\footnotesize
\begin{verbatim}
# Lists
## unordered list
* first item
* next item
    + sub-item
    + sub-item

## ordered list
1. first item
4. forgotten item
2. second item
2. third item
    + sub item
    + sub item

\end{verbatim}
  \end{itemize}
\end{frame}


\begin{frame}[fragile]\frametitle{Tables}
\footnotesize
\begin{verbatim}
Table Header 1 | Table Header 2
---------------|-----------------
cell 1|cell 2
cell 3|cell 4
\end{verbatim}
\end{frame}


\begin{frame}[fragile]\frametitle{Images}
\footnotesize
\begin{verbatim}
![R logo](img/Rlogo.jpg)
\end{verbatim}
\end{frame}


\begin{frame}[fragile]\frametitle{Exercise}
  Create a new Rmarkdown document with the title \emph{Wood-boring beetle} or something like this. We want to make a document containing some summary statistics and a logistic regression. The data set we wanna use is contained in the \texttt{asbio} package.
  \begin{enumerate}
  \item load the package and the data \texttt{beetle}
  \item there is a pdf with the name woodboring.pdf. Try to reproduce it!
    \begin{itemize}
    \item the first paragraph is taken from wikipedia (the link is contained in the document)
    \item the second from the help page of the data set
    \item use the Markdown Quick Reference in RStudio (via the Help menu)
    \item \url{https://www.rstudio.com/wp-content/uploads/2015/02/rmarkdown-cheatsheet.pdf}
    \item Find the error!
    \end{itemize}
  \end{enumerate}
\end{frame}


