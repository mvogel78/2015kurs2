% May 2015
% Autor: Mandy Vogel
% introduction

\documentclass[xcolor={table},c]{beamer}
%\usetheme[backgroundimagefile=mathe]{diepen}
\usetheme{Singapore}
% \useoutertheme{miniframes}

%\setbeamerfont{block title}{size=\small,series=\bfseries}
%\setbeamerfont{block body}{size=\footnotesize}

% \usecolortheme{beetle}
\usepackage{linkimage}

%\usepackage{handoutWithNotes}
%\pgfpagesuselayout{3 on 1 with notes}[a4paper,border shrink=5mm]

\begin{document}

\title{Introduction}   
\author{Mandy Vogel} 
\date{\today}

\AtBeginSection{
  \begin{frame}<beamer>[allowframebreaks,t]{Table of Contents}
    \tableofcontents[currentsection]
  \end{frame}}

\begin{frame}
\titlepage
\end{frame}

\begin{frame}[allowframebreaks,t]{Table of Contents}
\frametitle{Table of Contents}\tableofcontents
\end{frame}


\section{Reading Data}
\subsection{\texttt{read.table()}}
\begin{frame}\frametitle{Reading Data}
The most convenient way of reading data into R is via the function called
 \texttt{read.table()}. It requires that the data is in "ASCII format", or a "flat file" as
created with Windows' NotePad or any plain-text editor. The result of   \texttt{read.table()} is a
data frame.

\vspace*{0.5cm}

It is expected that each line of the data file corresponds to a subject information, that the
variables are separated by blanks or any other separator symbol (e.g., ",", ";"). The first
line of the file can contain a header (\texttt{header=T}) giving the names of the variables, which is highly
recommended!
\end{frame}

\begin{frame}[fragile]\frametitle{\texttt{read.table()}}
As an example we read in the data contained in the file \texttt{fishercats.txt} 
\begin{exampleblock}{Input/Output}\small
\begin{verbatim}
> read.table("session1data/fishercats.txt",  
+            sep=" ",header=T)
  Sex Bwt Hwt
1   F 2.0 7.0
2   F 2.0 7.4
3   F 2.0 9.5
4   F 2.1 7.2
5   F 2.1 7.3
....
\end{verbatim}
\end{exampleblock}
These data correspond to the heart and body weights of samples of male and female cats (R. A. Fisher, 1947).
\end{frame}

\begin{frame}[fragile]\frametitle{\texttt{read.table()}}
 The first argument corresponds to the data file, the second to the fields separator  and the third \texttt{header=T} specifies that the first line is a header with variable names. Important: the character variables will be automatically read as factors.

There is a variant for reading data from an url:
\begin{exampleblock}{Input/Output}\footnotesize
\begin{verbatim}
> winer <- read.table( 
+ "http://socserv.socsci.mcmaster.ca/jfox/Courses/R/ICPSR/Winer.txt",
+ header=T)
\end{verbatim}
\end{exampleblock}
\end{frame}

\begin{frame}[fragile]\frametitle{\texttt{read.table()}}
There are other variants of \texttt{read.table} function alike :
\begin{itemize}
\item \texttt{read.csv()} this function assumes that fields are separated by a comma instead of whites spaces
\item \texttt{read.csv2()} this function assumes that the separate symbol is the semicolon, but use a comma as the decimal point (some programs, e.g., Microsoft Excel, generate this format when running in European systems)
\item the function \texttt{scan()} is a powerful, but less friendly, way to read data in R; you may need it, if you want to read files with different numbers ov values per line
\end{itemize}
\end{frame}

\subsection{Other Sources}

\begin{frame}[fragile]\frametitle{Reading data from the clipboard}
With the function \texttt{read.delim()} or also \texttt{read.table()} it is possible to read data directly from the clipboard.

For example mark and copy some columns from an Excel spreadsheet and transfer this content to an R
by
\begin{exampleblock}{Input/Output}
\begin{verbatim}
 > mydata <- read.delim("clipboard",na.strings=".")
 > str(mydata) # structure of the data
\end{verbatim}
\end{exampleblock}
\end{frame}


\begin{frame}[fragile]\frametitle{The Data Editor}
To interactively edit a data frame in R you can use the edit function. For example:
\begin{exampleblock}{Input/Output}
\begin{verbatim}
> data(airquality)
> aq <-edit(airquality)
\end{verbatim}
\end{exampleblock}
This brings up a spreadsheet-like editor with a column for each variable in the data frame.
 See \texttt{help(airquality)}  for the contents of this data set.
The function \texttt{edit()} leaves the original data frame unchanged, the changed data frame is assigned to \texttt{aq}. The function \texttt{fix(x)} invokes the function \texttt{edit(x)} on \texttt{x} \textbf{and assign} the new (edited) version of \texttt{x} to \texttt{x}  
\end{frame}


\begin{frame}[fragile]\frametitle{Reading Data from Other Programs}
You can always use the export function from other (statistical) software to export data from other statistical systems to a tab or comma-delimited file and use the \texttt{read.table()}. However, R has some direct methods. 

The \texttt{foreign} package is one of the "recommended" packages in R. It contains routines to read files from SPSS (\texttt{.sav} format), SAS (export libraries), EpiInfo (.rec), Stata, Minitab, and some S-PLUS version 3 dump files. For example
\begin{exampleblock}{Input/Output}
\begin{verbatim}
> library(foreign)
> mydata <- read.spss("test.sav", to.data.frame=T)
\end{verbatim}
\end{exampleblock}
read the \texttt{test.sav} SPSS data set and convert it to a data.frame.
\end{frame}

\begin{frame}[fragile]\frametitle{Reading Data from Excel Files}
\begin{exampleblock}{Input/Output}
\small
\begin{verbatim}
> require(readxl)
Lade nötiges Paket: readxl
> content <- read_excel("data/Duncan.xls")
> head(content)
        Col0 type income education prestige
1 accountant prof     62        86       82
2      pilot prof     72        76       83
3  architect prof     75        92       90
4     author prof     55        90       76
5    chemist prof     64        86       90
6   minister prof     21        84       87
> 
\end{verbatim}
\end{exampleblock}
\end{frame}

\begin{frame}[fragile]\frametitle{Reading Data from Excel Files}
If someone is really fond of Excel, RExcel (http://rcom.univie.ac.at/download.html) is really worth the effort. There is also a function reading MSAccess files (\texttt{mdb.get()} from the Hmisc package)
\end{frame}

\begin{frame}\frametitle{Something on Connections}
The function \texttt{read.table()} opens a connection to a file, read the file, and close the connection. However, for data stored in databases, there exists a number of interface packages on CRAN. 

The RODBC package can set up ODBC connections to data stored by common applications including Excel and Access (for Excel and Access RODBC doesn't work on Unix but it is great for data base connections). There are also more general ways to build connections to data bases.

For up-to-date information on these matters, consult the "R Data Import/Export" manual that comes with the system.
\end{frame}

\section{Indexing/Subscripting}
\subsection{Indexing with Integers}
\begin{frame}[fragile]\frametitle{Indexing with Positive Integers} %%
\begin{itemize}
\item there are circumstances where we want to select only some of the elements of a vector/array/dataframe/list
\item this selection is done using subscripts (also known as indices)
\item subscripts have square brackets [2] while functions have round brackets (2)
\item Subscripts on vectors, matrices, arrays and dataframes have one set of square brackets [6], [3,4] or [2,3,2,1]
\item when a subscript appears as a blank it is understood to mean \emph{all of} thus
\begin{itemize}
\item \verb+[,4]+ means all rows in column 4 of an object
\item \verb+[2,]+ means all columns in row 2 of an object.
\item subscripts on lists have (usually) double square brackets [[2]] or [[i,j]]
\end{itemize}

\end{itemize}
\end{frame}

\begin{frame}[fragile,allowframebreaks]\frametitle{Indexing with Positive Integers}
\begin{itemize}
\item \emph{A vector of positive integers as index}:The index vector can be of any length and the result is of the same length as the index vector. For example,
\begin{exampleblock}{Input/Output}
\begin{verbatim}
> letters[1:3]
[1] "a" "b" "c"
> letters[c(1:3,1:3)]
[1] "a" "b" "c" "a" "b" "c"
>
\end{verbatim}
\end{exampleblock}
\end{itemize}
\end{frame}

\subsection{Logical Indexing}
\begin{frame}[fragile,allowframebreaks]\frametitle{Logical Indexing}
\begin{itemize}
\item \emph{A logical vector as index}: Values corresponding to T values in the index vector are selected and those corresponding to F or NA are omitted. For example,
\begin{exampleblock}{Input/Output}
\begin{verbatim}
> x<-c(1,2,3,NA)
> x[!is.na(x)]
[1] 1 2 3
\end{verbatim}
creates a vector without missing values. Also
\begin{verbatim}
> x[is.na(x)] <- 0
> x
[1] 1 2 3 0
\end{verbatim}
replaces the missing value by zeros.
\end{exampleblock}
\end{itemize}
\end{frame}


\begin{frame}[fragile]\frametitle{Logical Indexing} %%
A common operation is to select rows or columns of data frame that meet some criteria. For example, to select those rows of \texttt{painters} data frame with \texttt{Colour} $\geq$ 17:
\begin{exampleblock}{Input/Output}
\begin{verbatim}
> library(MASS)
> painters[painters$Colour >= 17,]
         Composition Drawing Colour Expression School
Bassano          6       8     17          0      D
Giorgione        8       9     18          4      D
Pordenone        8      14     17          5      D
...
\end{verbatim}
\end{exampleblock}
\end{frame}

\begin{frame}[fragile]\frametitle{Logical Indexing}
We may want to select on more than one criterion. We can combine logical indices by the 'and', 'or' and 'not' operators $\mathtt{\&,  | }$ and $\mathtt{!}$. For example,
\begin{exampleblock}{Input/Output}
\begin{verbatim}
> painters[painters$Colour >= 17 & 
                     painters$Composition > 10, c(1,2,3)]
          Composition Drawing Colour
Titian             12      15     18
Rembrandt          15       6     17
Rubens             18      13     17
Van Dyck           15      10     17
\end{verbatim}
\end{exampleblock}
\end{frame}


\begin{frame}[fragile]\frametitle{List of Basic Logical Operations}
    \rowcolors[]{1}{gray!10}{gray!30}
  \begin{tabular}{@{} >{\ttfamily}l p{7cm}} 
    \rowcolor{gray!40}
    \textbf{Operation} & \textbf{Description}\\
$!$ &        logical NOT                         \\
$\&$ &       logical AND                         \\
$|$ &       logical OR                          \\
$<$ &        less than                           \\
$<=$ &       less than or equal to               \\
$>$ &        greater than                        \\
$>=$ &       greater than or equal to            \\
$==$ &       logical equals (double =)           \\
$!=$ &       not equal                           \\
isTRUE(x) & an abbreviation of identical(TRUE,x)\\
\end{tabular}
\end{frame}


\begin{frame}[fragile]\frametitle{Logical Indexing}
If we want to select a subgroup, for example those with schools A, B, and D. We can generate
 a logical vector using the  $\mathtt{\%in\%}$ operator as follows:
\begin{exampleblock}{Input/Output}\small
\begin{verbatim}
> painters[painters$School %in% c("A","C","D"),]
Da Udine           10       8     16        3      A
Da Vinci           15      16      4       14      A
Del Piombo          8      13     16        7      A
...
\end{verbatim}
\end{exampleblock}
\end{frame}

\begin{frame}[fragile]\frametitle{Logical Indexing}
Sometimes we are interested in the indices of rows satisfying a certain condition. To extract these indices we use the \texttt{which()} command.
\begin{exampleblock}{Input/Output}\small
\begin{verbatim}
> which(painters$School %in% c("A","C","D"))
 [1]  1  2  3  4  5  6  7  8  9 10 17 18 19 20 21 22 23 24 25 26 27 28 29 30 31
[26] 32
\end{verbatim}
\end{exampleblock}
\end{frame}


\subsection{Indexing with Characters}
\begin{frame}[fragile]\frametitle{Indexing}
A vector character strings with variable names can be used to extract those variables relevant for analysis. This is very useful when we have a large number of variables and we need to work with a few ones. For example,
\begin{exampleblock}{Input/Output}\small
\begin{verbatim}
> names(painters)
[1] "Composition" "Drawing" "Colour" "Expression" "School"
> painters[1:3,c("Drawing","Expression")]
           Drawing Expression
Da Udine         8          3
Da Vinci        16         14
Del Piombo      13          7
\end{verbatim}
\end{exampleblock}
\end{frame}



\begin{frame}[fragile]\frametitle{Indexing with Characters} %%
\begin{itemize}
\item \emph{a vector of character strings} could a index on a vector when the vector has names:
  \begin{exampleblock}{Input/Output}
\begin{verbatim}
> x <- c(1:3,NA)
> names(x)<-letters[1:4]
> x
 a  b  c  d
 1  2  3 NA
> x[c("a","c")]
a c
1 3
\end{verbatim}
  \end{exampleblock}
\end{itemize}
\end{frame}

\subsection{Indexing with Negative Indices}
\begin{frame}[fragile]\frametitle{Trimming Vectors Using Negative Indices} %%
\begin{itemize}
\item an extremely useful facility is to use negative indices to drop terms from a vector
\item suppose we wanted a new vector, z, to contain everything but the first element of x
  \begin{exampleblock}{Input/Output}
\begin{verbatim} 
> x<- c(5,8,6,7,1,5,3)
> (z <- x[-1])
[1] 8 6 7 1 5 3
\end{verbatim}
  \end{exampleblock}
\end{itemize}
\end{frame}



\end{document}
